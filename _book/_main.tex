% Options for packages loaded elsewhere
\PassOptionsToPackage{unicode}{hyperref}
\PassOptionsToPackage{hyphens}{url}
%
\documentclass[
]{book}
\usepackage{amsmath,amssymb}
\usepackage{iftex}
\ifPDFTeX
  \usepackage[T1]{fontenc}
  \usepackage[utf8]{inputenc}
  \usepackage{textcomp} % provide euro and other symbols
\else % if luatex or xetex
  \usepackage{unicode-math} % this also loads fontspec
  \defaultfontfeatures{Scale=MatchLowercase}
  \defaultfontfeatures[\rmfamily]{Ligatures=TeX,Scale=1}
\fi
\usepackage{lmodern}
\ifPDFTeX\else
  % xetex/luatex font selection
\fi
% Use upquote if available, for straight quotes in verbatim environments
\IfFileExists{upquote.sty}{\usepackage{upquote}}{}
\IfFileExists{microtype.sty}{% use microtype if available
  \usepackage[]{microtype}
  \UseMicrotypeSet[protrusion]{basicmath} % disable protrusion for tt fonts
}{}
\makeatletter
\@ifundefined{KOMAClassName}{% if non-KOMA class
  \IfFileExists{parskip.sty}{%
    \usepackage{parskip}
  }{% else
    \setlength{\parindent}{0pt}
    \setlength{\parskip}{6pt plus 2pt minus 1pt}}
}{% if KOMA class
  \KOMAoptions{parskip=half}}
\makeatother
\usepackage{xcolor}
\usepackage{color}
\usepackage{fancyvrb}
\newcommand{\VerbBar}{|}
\newcommand{\VERB}{\Verb[commandchars=\\\{\}]}
\DefineVerbatimEnvironment{Highlighting}{Verbatim}{commandchars=\\\{\}}
% Add ',fontsize=\small' for more characters per line
\usepackage{framed}
\definecolor{shadecolor}{RGB}{248,248,248}
\newenvironment{Shaded}{\begin{snugshade}}{\end{snugshade}}
\newcommand{\AlertTok}[1]{\textcolor[rgb]{0.94,0.16,0.16}{#1}}
\newcommand{\AnnotationTok}[1]{\textcolor[rgb]{0.56,0.35,0.01}{\textbf{\textit{#1}}}}
\newcommand{\AttributeTok}[1]{\textcolor[rgb]{0.13,0.29,0.53}{#1}}
\newcommand{\BaseNTok}[1]{\textcolor[rgb]{0.00,0.00,0.81}{#1}}
\newcommand{\BuiltInTok}[1]{#1}
\newcommand{\CharTok}[1]{\textcolor[rgb]{0.31,0.60,0.02}{#1}}
\newcommand{\CommentTok}[1]{\textcolor[rgb]{0.56,0.35,0.01}{\textit{#1}}}
\newcommand{\CommentVarTok}[1]{\textcolor[rgb]{0.56,0.35,0.01}{\textbf{\textit{#1}}}}
\newcommand{\ConstantTok}[1]{\textcolor[rgb]{0.56,0.35,0.01}{#1}}
\newcommand{\ControlFlowTok}[1]{\textcolor[rgb]{0.13,0.29,0.53}{\textbf{#1}}}
\newcommand{\DataTypeTok}[1]{\textcolor[rgb]{0.13,0.29,0.53}{#1}}
\newcommand{\DecValTok}[1]{\textcolor[rgb]{0.00,0.00,0.81}{#1}}
\newcommand{\DocumentationTok}[1]{\textcolor[rgb]{0.56,0.35,0.01}{\textbf{\textit{#1}}}}
\newcommand{\ErrorTok}[1]{\textcolor[rgb]{0.64,0.00,0.00}{\textbf{#1}}}
\newcommand{\ExtensionTok}[1]{#1}
\newcommand{\FloatTok}[1]{\textcolor[rgb]{0.00,0.00,0.81}{#1}}
\newcommand{\FunctionTok}[1]{\textcolor[rgb]{0.13,0.29,0.53}{\textbf{#1}}}
\newcommand{\ImportTok}[1]{#1}
\newcommand{\InformationTok}[1]{\textcolor[rgb]{0.56,0.35,0.01}{\textbf{\textit{#1}}}}
\newcommand{\KeywordTok}[1]{\textcolor[rgb]{0.13,0.29,0.53}{\textbf{#1}}}
\newcommand{\NormalTok}[1]{#1}
\newcommand{\OperatorTok}[1]{\textcolor[rgb]{0.81,0.36,0.00}{\textbf{#1}}}
\newcommand{\OtherTok}[1]{\textcolor[rgb]{0.56,0.35,0.01}{#1}}
\newcommand{\PreprocessorTok}[1]{\textcolor[rgb]{0.56,0.35,0.01}{\textit{#1}}}
\newcommand{\RegionMarkerTok}[1]{#1}
\newcommand{\SpecialCharTok}[1]{\textcolor[rgb]{0.81,0.36,0.00}{\textbf{#1}}}
\newcommand{\SpecialStringTok}[1]{\textcolor[rgb]{0.31,0.60,0.02}{#1}}
\newcommand{\StringTok}[1]{\textcolor[rgb]{0.31,0.60,0.02}{#1}}
\newcommand{\VariableTok}[1]{\textcolor[rgb]{0.00,0.00,0.00}{#1}}
\newcommand{\VerbatimStringTok}[1]{\textcolor[rgb]{0.31,0.60,0.02}{#1}}
\newcommand{\WarningTok}[1]{\textcolor[rgb]{0.56,0.35,0.01}{\textbf{\textit{#1}}}}
\usepackage{longtable,booktabs,array}
\usepackage{calc} % for calculating minipage widths
% Correct order of tables after \paragraph or \subparagraph
\usepackage{etoolbox}
\makeatletter
\patchcmd\longtable{\par}{\if@noskipsec\mbox{}\fi\par}{}{}
\makeatother
% Allow footnotes in longtable head/foot
\IfFileExists{footnotehyper.sty}{\usepackage{footnotehyper}}{\usepackage{footnote}}
\makesavenoteenv{longtable}
\usepackage{graphicx}
\makeatletter
\newsavebox\pandoc@box
\newcommand*\pandocbounded[1]{% scales image to fit in text height/width
  \sbox\pandoc@box{#1}%
  \Gscale@div\@tempa{\textheight}{\dimexpr\ht\pandoc@box+\dp\pandoc@box\relax}%
  \Gscale@div\@tempb{\linewidth}{\wd\pandoc@box}%
  \ifdim\@tempb\p@<\@tempa\p@\let\@tempa\@tempb\fi% select the smaller of both
  \ifdim\@tempa\p@<\p@\scalebox{\@tempa}{\usebox\pandoc@box}%
  \else\usebox{\pandoc@box}%
  \fi%
}
% Set default figure placement to htbp
\def\fps@figure{htbp}
\makeatother
\setlength{\emergencystretch}{3em} % prevent overfull lines
\providecommand{\tightlist}{%
  \setlength{\itemsep}{0pt}\setlength{\parskip}{0pt}}
\setcounter{secnumdepth}{5}
\usepackage{booktabs}
\usepackage{booktabs}
\usepackage{longtable}
\usepackage{array}
\usepackage{multirow}
\usepackage{wrapfig}
\usepackage{float}
\usepackage{colortbl}
\usepackage{pdflscape}
\usepackage{tabu}
\usepackage{threeparttable}
\usepackage{threeparttablex}
\usepackage[normalem]{ulem}
\usepackage{makecell}
\usepackage{xcolor}
\usepackage[]{natbib}
\bibliographystyle{plainnat}
\usepackage{bookmark}
\IfFileExists{xurl.sty}{\usepackage{xurl}}{} % add URL line breaks if available
\urlstyle{same}
\hypersetup{
  pdftitle={Viet Nam Household Living Standards Survey Codebook},
  pdfauthor={Development and Policies Research Center (DEPOCEN)},
  hidelinks,
  pdfcreator={LaTeX via pandoc}}

\title{Viet Nam Household Living Standards Survey Codebook}
\author{Development and Policies Research Center (DEPOCEN)}
\date{2026-01-10}

\begin{document}
\maketitle

{
\setcounter{tocdepth}{1}
\tableofcontents
}
\chapter*{Preface}\label{preface}
\addcontentsline{toc}{chapter}{Preface}

The Viet Nam Household Living Standards Survey (VHLSS) is one of the most comprehensive and widely used micro-datasets for socio-economic research in Viet Nam. However, as the survey has evolved significantly since its inception in 1993 through changes in sampling design, questionnaire structure, and variable naming, longitudinal analysis requires rigorous data cleaning and harmonization.

Through our work with this dataset, we noticed that while many researchers have cleaned and used the VHLSS, most work in isolation. This leads to inconsistent cleaning procedures and a narrow focus on specific modules. We encountered this same fragmentation even within our own team, and it is a challenge shared by the broader research community. Thus, this documentation was developed by the Development and Policies Research Center (DEPOCEN) to serve as a standardized guide for researchers and data analysts. Our goal is to ensure that the data cleaning process is:

\begin{itemize}
\tightlist
\item
  Reproducible: All steps are scripted to ensure consistency.
\item
  Transparent: Every decision regarding outliers, missing values, and variable merging is documented.
\item
  Accessible: By centralizing metadata and harmonization crosswalks, we aim to reduce the ``entry barrier'' for new researchers using VHLSS.
\end{itemize}

\textbf{Who are we?}

This codebook was developed by a research team at DEPOCEN, comprising \textbf{Tuong-Vy Phan, Huy Le Vu, and Nguyen Thi Hong Tram}, under the supervision of \textbf{Dr.~Anh Ngoc Nguyen}. We would like to express their gratitude to \textbf{Dr.~Doan Quang Hung} for providing the core variable list and to \textbf{Dr.~Vu Hoang Linh} for sharing the data-cleaning scripts for the consumption module.

We are grateful for the comments and support of other DEPOCEN researchers, including Manh-Duc Doan, Quang-Thanh Tran, Nguyen Viet Lien, and Ha-My Bui.

\textbf{Who is this for?}

This guide is intended for economists, policy analysts, and students who are working with VHLSS microdata. We assume a basic understanding of statistical software (specifically Stata and R) and familiarity with household survey structures.

\textbf{Structure of the Documentation}

\begin{itemize}
\item
  Data Structure: An overview of the VHLSS history and sample design.
\item
  Cleaning Procedure: A step-by-step workflow for moving from raw files to a master harmonized dataset, with provided Stata code.
\item
  Notes \& References: A collection of insights from the wider economic research community.
\end{itemize}

\textbf{Acknowledgements}

We would like to thank the \href{https://www.nso.gov.vn/en/homepage/}{General Statistics Office (GSO)} for their work in conducting these surveys and the various scholars whose previous notes on VHLSS provided the foundation for this consolidated procedure.

\chapter{VHLSS Introduction}\label{vhlss-introduction}

\section*{Purpose}\label{purpose}
\addcontentsline{toc}{section}{Purpose}

To evaluate living standards for policy-making and socio-economic development planning, from 1993 to now the General Statistics Office (GSO) conducts the Viet Nam Household Living Standards Survey (VHLSS). The purpose of the VHLSS in order to systematically monitor and supervise the living standards of different population groups in Viet Nam; to monitor and evaluate the implementation of the Comprehensive Poverty Reduction and Growth Strategy; and to contribute to the evaluation of achievement of the Sustainable Development Goals (SDGs) and Vietnam's socio-economic development goals.

\section*{Availability}\label{availability}
\addcontentsline{toc}{section}{Availability}

From 2002 to 2010, this survey has been conducted regularly by the GSO every two years. From 2011 to 2022, VHLSS are conducted annually. However, the odd-numbered year surveys only collect data on demographics, employment and income.

\section*{Survey period}\label{survey-period}
\addcontentsline{toc}{section}{Survey period}

\begin{itemize}
\tightlist
\item
  The survey was conducted in four periods in March, June, September and December. The period for collecting information in the locality is one month.
\item
  The reference period of household income and expenditure was the last 12 months.
\end{itemize}

\section*{Coverage of the survey}\label{coverage-of-the-survey}
\addcontentsline{toc}{section}{Coverage of the survey}

Geographically, the survey covered the whole country. Scope of the survey included all selected enumeration areas and communes in 63 provinces and cities under central management.

\section*{Data collection method}\label{data-collection-method}
\addcontentsline{toc}{section}{Data collection method}

Face-to-face interviews

\chapter{Dataset Basic Information}\label{dataset-basic-information}

\section{Outline of the survey}\label{outline-of-the-survey}

\begin{itemize}
\tightlist
\item
  Section 1. Basic demographic characteristics related to living standards
\item
  Section 2. Education
\item
  Section 3. Labor - Employment
\item
  Section 4. Health and health care
\item
  Section 5. Income
\item
  Section 6. Consumption expenditure
\item
  Section 7. Durable goods
\item
  Section 8. Housing, electricity, water, sanitation facilities and use of Internet
\item
  Section 9. Participation in poverty reduction programs
\item
  Section 10. Business production activities
\item
  Section 11. Commune general characteristics
\end{itemize}

\section{Sample Size}\label{sample-size}

\textbf{Sample Overview}

\begin{table}[!h]
\centering
\caption{\label{tab:unnamed-chunk-1}Sample Size by Survey Year (households)}
\centering
\begin{tabular}[t]{>{\raggedright\arraybackslash}p{10em}>{\raggedright\arraybackslash}p{12em}>{\raggedright\arraybackslash}p{12em}>{\raggedright\arraybackslash}p{12em}}
\toprule
\cellcolor[HTML]{1a2980}{\textcolor{white}{\textbf{Year}}} & \cellcolor[HTML]{1a2980}{\textcolor{white}{\textbf{Total}}} & \cellcolor[HTML]{1a2980}{\textcolor{white}{\textbf{Income}}} & \cellcolor[HTML]{1a2980}{\textcolor{white}{\textbf{Expenditure}}}\\
\midrule
\ttfamily{\cellcolor{gray!10}{2008}} & \cellcolor{gray!10}{45.945} & \cellcolor{gray!10}{} & \cellcolor{gray!10}{9.189}\\
\ttfamily{2010} & 69.360 &  & 9.399\\
\ttfamily{\cellcolor{gray!10}{2012}} & \cellcolor{gray!10}{} & \cellcolor{gray!10}{} & \cellcolor{gray!10}{9.399}\\
\ttfamily{2014} & 46.995 & 37.596 & 9.399\\
\ttfamily{\cellcolor{gray!10}{2016}} & \cellcolor{gray!10}{46.995} & \cellcolor{gray!10}{37.596} & \cellcolor{gray!10}{9.399}\\
\addlinespace
\ttfamily{2018} & 46.995 & 37.596 & 9.399\\
\ttfamily{\cellcolor{gray!10}{2020}} & \cellcolor{gray!10}{46.980} & \cellcolor{gray!10}{} & \cellcolor{gray!10}{}\\
\bottomrule
\multicolumn{4}{l}{\rule{0pt}{1em}\underline{\textit{Notes:}} Total: Total surveyed households; Income: Households were asked about income and other issues; Expenditure: Households were asked about income, expenditure and other issues}\\
\end{tabular}
\end{table}

The sample for the the year \(t\) Residential Living Standards Survey (KSMS \(t\)) was designed in 2 steps as follows:

\textbf{Step 1: Select Survey Areas}

Select 3,133 areas from the master sample, structured as follows:

\begin{itemize}
\tightlist
\item
  25\% (≈ 783 areas): Re-selected from areas surveyed only in KSMS \(t-2\)
\item
  25\% (≈ 783 areas): Re-selected from areas surveyed in both KSMS \(t-2\) and KSMS \(t-1\)
\item
  25\% (≈ 783 areas): Re-selected from areas surveyed only in KSMS \(t-1\)
\item
  25\% (≈ 783 areas): Newly selected from the master sample
\end{itemize}

\textbf{Step 2: Select Households for Survey}

For areas re-selected from KSMS \(t-1\)/\(t-2\):

\begin{itemize}
\tightlist
\item
  Select all 15 households previously surveyed in \(t-2\) and/or \(t-1\)
\item
  If a household is no longer in the area, a replacement household will be chosen
\item
  Additionally select 5 reserve households from the reserve lists of previous years (if insufficient, select adjacent households)
\end{itemize}

For newly selected areas:

\begin{itemize}
\tightlist
\item
  Update the list of all households in the area
\item
  Select 20 households using the systematic random method from the updated list
\item
  From these, select 15 main households and 5 reserve households
\end{itemize}

\section{Rotating panel}\label{rotating-panel}

The VHLSS uses a rotating panel design, allowing for the construction of a panel across three survey rounds (e.g., 2014-2016-2018), provided the data is based on the same Population and Housing Census sampling frame. For instance, the household systems from 2010 to 2018 were designed using the 2009 Census. However, since the 2020 VHLSS was based on the 2019 Census, it is not possible to create a panel linking VHLSS 2016-2018 with 2020.

\section{Weight Data}\label{weight-data}

At household level, each VHLSS will have a weihgt data for each year. Usually we would use \(wt9\), which is the weight of 9,000 household. There are also \(wt36\) and \(wt45\) is for file with 36,000 or 45,000 household.

At individual level, we need to take the weight divided by the number of family member.
\(\text{Weight individual} = \frac{\text{Weight household}}{\text{Houshold size}}\)

\chapter{Data Processing Procedure}\label{data-processing-procedure}

Analyzing the VHLSS longitudinally presents challenges due to structural design changes, administrative adjustments, and varying module contents. This section outlines these inconsistencies and our strategies for data harmonization. For this version, we utilize the 9,000-household sample to construct a pooled individual-level dataset, focusing on a specific subset of variables.

The following sections provide a step-by-step guide to navigating this documentation.

\section{Folder organization}\label{folder-organization}

The folder contains data, do file and other materials are organized as follows:

\begin{table}[!h]
\centering
\begin{tabular}{>{}lll}
\toprule
\textbf{Sub-folder} & \textbf{Description} & \textbf{Action}\\
\midrule
\ttfamily{\cellcolor{gray!10}{01\_dofiles}} & \cellcolor{gray!10}{Code in Stata to clean the datasets} & \cellcolor{gray!10}{Run master.do only to modify cleaning}\\
\ttfamily{02\_data} & Raw dataset in Stata (.dta) format & Do not modify\\
\ttfamily{\cellcolor{gray!10}{03\_temp}} & \cellcolor{gray!10}{Temporary files} & \cellcolor{gray!10}{Do not modify}\\
\ttfamily{04\_clean} & Cleaned data in Stata (.dta) format & Download and use\\
\ttfamily{\cellcolor{gray!10}{05\_metadata}} & \cellcolor{gray!10}{Questionnaire and codebook} & \cellcolor{gray!10}{Access questionnaire and codebook here}\\
\bottomrule
\end{tabular}
\end{table}

After organizing folder, first thing we need to do is clear all settings and set working directory.

\section{Clear all settings}\label{clear-all-settings}

\begin{Shaded}
\begin{Highlighting}[]
\NormalTok{cap }\FunctionTok{log} \KeywordTok{close}
\KeywordTok{clear} \OtherTok{all}
\KeywordTok{clear} \FunctionTok{matrix}
\KeywordTok{set} \KeywordTok{more} \KeywordTok{off}
\NormalTok{eststo }\KeywordTok{clear}
\end{Highlighting}
\end{Shaded}

\section{Set Working Directory}\label{set-working-directory}

Replace username and path to your folder.

\begin{Shaded}
\begin{Highlighting}[]
\KeywordTok{if} \StringTok{"\textasciigrave{}c(username)\textquotesingle{}"}\NormalTok{ == }\StringTok{"XXX"}\NormalTok{ \{}
\NormalTok{gl MyProject }\StringTok{"/Users/XXX/My Drive/DEPOCEN {-} VHLSS Data cleaning"}
\NormalTok{    gl }\KeywordTok{data} \StringTok{"$MyProject/02\_data"}
\NormalTok{    gl temp  }\StringTok{"$MyProject/03\_temp"}
\NormalTok{    gl }\KeywordTok{clean} \StringTok{"$MyProject/04\_clean"}
\NormalTok{    gl metadata }\StringTok{"$MyProject/05\_metadata"}
\NormalTok{\}}
\end{Highlighting}
\end{Shaded}

\section{Clean data for each year}\label{clean-data-for-each-year}

\subsection*{Make sure each file is uniquely defined}\label{make-sure-each-file-is-uniquely-defined}
\addcontentsline{toc}{subsection}{Make sure each file is uniquely defined}

As each survey year consists of multiple section-specific files, we must ensure all observations are uniquely identified prior to merging.

\begin{Shaded}
\begin{Highlighting}[]
\NormalTok{forval i = 14 (2) 18 \{}
    \KeywordTok{foreach} \FunctionTok{m} \KeywordTok{in}\NormalTok{ Muc1A Muc2A Muc2X Muc3A Muc3C Muc4a \{}
        \KeywordTok{local}\NormalTok{ filepath }\StringTok{"$data/vhlss\_20\textasciigrave{}i\textquotesingle{}/hh\_9000/\textasciigrave{}m\textquotesingle{}.dta"}
        \KeywordTok{capture} \KeywordTok{confirm}\NormalTok{ file }\StringTok{"\textasciigrave{}filepath\textquotesingle{}"}
        \KeywordTok{if} \DataTypeTok{\_rc}\NormalTok{ == 0 \{}
            \KeywordTok{use} \StringTok{"\textasciigrave{}filepath\textquotesingle{}"}\NormalTok{, }\KeywordTok{clear}
            \KeywordTok{if}\NormalTok{ \_N \textgreater{} 0 \{}
                \KeywordTok{duplicates} \KeywordTok{drop}\NormalTok{ tinh huyen xa diaban hoso matv, }\KeywordTok{force}
                \KeywordTok{tempfile}\NormalTok{ uniq\_}\OtherTok{\textasciigrave{}m\textquotesingle{}}\NormalTok{\_20}\OtherTok{\textasciigrave{}i\textquotesingle{}}
                \KeywordTok{save} \OtherTok{\textasciigrave{}uniq\_\textasciigrave{}m\textquotesingle{}}\NormalTok{\_20}\OtherTok{\textasciigrave{}i\textquotesingle{}}\NormalTok{\textquotesingle{}, }\KeywordTok{replace}
\NormalTok{            \}}
\NormalTok{        \}}
\NormalTok{    \}}
\NormalTok{\}}
\end{Highlighting}
\end{Shaded}

\subsection*{Merge individual file}\label{merge-individual-file}
\addcontentsline{toc}{subsection}{Merge individual file}

\begin{Shaded}
\begin{Highlighting}[]
\NormalTok{forval i = 14 (2) 18 \{}
    \KeywordTok{use} \OtherTok{\textasciigrave{}uniq\_Muc1A\_20\textasciigrave{}i\textquotesingle{}}\NormalTok{\textquotesingle{}, }\KeywordTok{clear}
        \KeywordTok{foreach} \FunctionTok{m} \KeywordTok{in}\NormalTok{ Muc2A Muc2X Muc3A Muc3C Muc4a \{}
            \KeywordTok{capture} \KeywordTok{merge}\NormalTok{ 1:1 tinh huyen xa diaban hoso matv }\KeywordTok{using} \OtherTok{\textasciigrave{}uniq\_\textasciigrave{}m\textquotesingle{}}\NormalTok{\_20}\OtherTok{\textasciigrave{}i\textquotesingle{}}\NormalTok{\textquotesingle{}}
            \KeywordTok{capture} \KeywordTok{drop} \KeywordTok{if}\NormalTok{ \_m == 2}
            \KeywordTok{capture} \KeywordTok{drop}\NormalTok{ \_m}
\NormalTok{        \}}
    \KeywordTok{save} \StringTok{"$temp/individual\_20\textasciigrave{}i\textquotesingle{}.dta"}\NormalTok{, }\KeywordTok{replace}
\NormalTok{\}}
\end{Highlighting}
\end{Shaded}

\subsection*{Extract variables from household file}\label{extract-variables-from-household-file}
\addcontentsline{toc}{subsection}{Extract variables from household file}

These variables do not appear in individual file, so we need to take them from household file.

\begin{Shaded}
\begin{Highlighting}[]
\NormalTok{forval i = 14 (2) 18 \{}
    \KeywordTok{use} \StringTok{"$data/vhlss\_20\textasciigrave{}i\textquotesingle{}/hh\_9000/Ho1.dta"}\NormalTok{, }\KeywordTok{clear}
    \KeywordTok{keep}\NormalTok{ tinh huyen xa diaban hoso ttnt dantoc tsnguoi }\CommentTok{// take area, ethnic and household size variable}
    \KeywordTok{duplicates} \KeywordTok{drop}\NormalTok{ tinh huyen xa diaban hoso, }\KeywordTok{force}
    
    \KeywordTok{tempfile}\NormalTok{ indiv1\_}\OtherTok{\textasciigrave{}i\textquotesingle{}}
    \KeywordTok{save} \OtherTok{\textasciigrave{}indiv1\_\textasciigrave{}i\textquotesingle{}}\NormalTok{\textquotesingle{}}
    
    \KeywordTok{use} \StringTok{"$data/vhlss\_20\textasciigrave{}i\textquotesingle{}/hh\_9000/Muc5a1.dta"}\NormalTok{, }\KeywordTok{clear}
    \KeywordTok{keep}\NormalTok{ tinh huyen xa diaban hoso m5a1ma m5a1c2a m5a1c2b m5a1c3a m5a1c3b }\CommentTok{// take food consumption variabls}
    \KeywordTok{duplicates} \KeywordTok{drop}\NormalTok{ tinh huyen xa diaban hoso, }\KeywordTok{force}
    
    \KeywordTok{tempfile}\NormalTok{ indiv2\_}\OtherTok{\textasciigrave{}i\textquotesingle{}}
    \KeywordTok{save} \OtherTok{\textasciigrave{}indiv2\_\textasciigrave{}i\textquotesingle{}}\NormalTok{\textquotesingle{}}
    
    \KeywordTok{use} \StringTok{"$data/vhlss\_20\textasciigrave{}i\textquotesingle{}/hh\_9000/Muc5a2.dta"}\NormalTok{, }\KeywordTok{clear}
    \KeywordTok{keep}\NormalTok{ tinh huyen xa diaban hoso m5a2ma m5a2c2a m5a2c2b }\CommentTok{// take food consumption variabls}
    \KeywordTok{duplicates} \KeywordTok{drop}\NormalTok{ tinh huyen xa diaban hoso, }\KeywordTok{force}
    \KeywordTok{destring}\NormalTok{ m5a2ma, }\KeywordTok{replace}
    
    \KeywordTok{tempfile}\NormalTok{ indiv3\_}\OtherTok{\textasciigrave{}i\textquotesingle{}}
    \KeywordTok{save} \OtherTok{\textasciigrave{}indiv3\_\textasciigrave{}i\textquotesingle{}}\NormalTok{\textquotesingle{}}
    
    \KeywordTok{use} \StringTok{"$temp/individual\_20\textasciigrave{}i\textquotesingle{}.dta"}\NormalTok{, }\KeywordTok{clear}
\NormalTok{        forval j = 1/3 \{}
            \KeywordTok{merge} \FunctionTok{m}\NormalTok{:1 tinh huyen xa diaban hoso }\KeywordTok{using} \OtherTok{\textasciigrave{}indiv\textasciigrave{}j\textquotesingle{}}\NormalTok{\_}\OtherTok{\textasciigrave{}i\textquotesingle{}}\NormalTok{\textquotesingle{}}
            \KeywordTok{drop} \KeywordTok{if}\NormalTok{ \_m == 2}
            \KeywordTok{drop}\NormalTok{ \_m}
\NormalTok{        \}}
    \KeywordTok{save} \StringTok{"$temp/individual\_20\textasciigrave{}i\textquotesingle{}.dta"}\NormalTok{, }\KeywordTok{replace}
\NormalTok{\}}
\end{Highlighting}
\end{Shaded}

\subsection*{Label define all variables}\label{label-define-all-variables}
\addcontentsline{toc}{subsection}{Label define all variables}

Due to the legacy GSO encoding used in the raw data, we converted the character sets into a readable format. Specifically, we extracted the correct value labels for all variables of interest into a separate do-file, which is then executed within the master script.

\begin{Shaded}
\begin{Highlighting}[]
\NormalTok{forval i = 14 (2) 18 \{}
    \KeywordTok{use} \StringTok{"$temp/individual\_20\textasciigrave{}i\textquotesingle{}.dta"}\NormalTok{, }\KeywordTok{clear}
    
    \CommentTok{// Run the fixed labels}
    \KeywordTok{do} \StringTok{"$temp/vhlss\_labels\_fixed.do"}
    
    \CommentTok{// Automatically attach labels to variables with the same name}
    \KeywordTok{foreach}\NormalTok{ v }\KeywordTok{of} \KeywordTok{varlist} \DataTypeTok{\_all}\NormalTok{ \{}
        \KeywordTok{capture} \KeywordTok{label} \KeywordTok{values} \OtherTok{\textasciigrave{}v\textquotesingle{}} \OtherTok{\textasciigrave{}v\textquotesingle{}}
\NormalTok{    \}}
    
    \KeywordTok{save} \StringTok{"$temp/individual\_20\textasciigrave{}i\textquotesingle{}.dta"}\NormalTok{, }\KeywordTok{replace}
\NormalTok{\}}
\end{Highlighting}
\end{Shaded}

\section{Append individual dataset}\label{append-individual-dataset}

\subsection*{Import the meta data}\label{import-the-meta-data}
\addcontentsline{toc}{subsection}{Import the meta data}

Variable names and the ordering of questions frequently change between survey waves. For instance, the variable for \texttt{education} may be coded as \emph{m2ac2a} in one year and \emph{m2xc2a} in another. To facilitate analysis, we have compiled comprehensive metadata mapping variable names and labels for each year, which is available in the \texttt{05\_metadata/VHLSS\_codebook\_9k\_thanhvien\ -\ thanhvien.csv}. This dataset is restricted to variables of interest. To add further variables, download the \href{https://docs.google.com/spreadsheets/d/1zyGJARgK3_1daXMCwZtn-5dU-kDYZ0_u3rRJUfMP8xQ/edit?usp=sharing}{metadata} and ensure all new variables are harmonized to account for naming inconsistencies across years.

\begin{Shaded}
\begin{Highlighting}[]
\NormalTok{import delimited }\StringTok{"$metadata/VHLSS\_codebook\_9k\_thanhvien {-} thanhvien.csv"}\NormalTok{, varnames(1) encoding(UTF{-}8) }\KeywordTok{clear}
\end{Highlighting}
\end{Shaded}

\subsection*{Rename and label code}\label{rename-and-label-code}
\addcontentsline{toc}{subsection}{Rename and label code}

\begin{Shaded}
\begin{Highlighting}[]
\KeywordTok{local} \KeywordTok{N}\NormalTok{ = \_N}

\NormalTok{forval i = 2014 (2) 2018 \{}
    \CommentTok{// Open file}
\NormalTok{    file open myfile }\KeywordTok{using} \StringTok{\textasciigrave{}"}\OtherTok{$temp}\NormalTok{/label\_in}\OtherTok{\textasciigrave{}i\textquotesingle{}}\NormalTok{.do}\StringTok{"\textquotesingle{}, write text replace}
\StringTok{    }
\StringTok{    local renamed\_vars ""}
\StringTok{    forval iii = 1/\textasciigrave{}N\textquotesingle{} \{}
\StringTok{        local vvvcode = code[\textasciigrave{}iii\textquotesingle{}]}
\StringTok{        local vvv\_i = code\_\textasciigrave{}i\textquotesingle{}[\textasciigrave{}iii\textquotesingle{}]}
\StringTok{        local v\_desc = description[\textasciigrave{}iii\textquotesingle{}]}
\StringTok{        }
\StringTok{        // Check if code\_\textasciigrave{}i\textquotesingle{} is not empty}
\StringTok{        if \textasciigrave{}"}\OtherTok{\textasciigrave{}vvv\_i\textquotesingle{}}\StringTok{"\textquotesingle{} != "" \{}
\StringTok{                // 1. Create the Rename command}
\StringTok{            local result = \textasciigrave{}"}\NormalTok{ren }\OtherTok{\textasciigrave{}vvv\_i\textquotesingle{}} \OtherTok{\textasciigrave{}vvvcode\textquotesingle{}}\StringTok{"\textquotesingle{}}
\StringTok{            file write myfile \textasciigrave{}"}\OtherTok{\textasciigrave{}result\textquotesingle{}}\StringTok{"\textquotesingle{} \_n}
\StringTok{                }
\StringTok{                // 2. Create the Label command}
\StringTok{            local lab\_cmd \textasciigrave{}"}\NormalTok{label }\KeywordTok{variable} \OtherTok{\textasciigrave{}vvvcode\textquotesingle{}} \StringTok{\textasciigrave{}"}\OtherTok{\textasciigrave{}v\_desc\textquotesingle{}}\StringTok{"\textquotesingle{}"}\NormalTok{\textquotesingle{}}
\NormalTok{            file write myfile }\StringTok{\textasciigrave{}"}\OtherTok{\textasciigrave{}lab\_cmd\textquotesingle{}}\StringTok{"\textquotesingle{} \_n}
\StringTok{            }
\StringTok{            local renamed\_vars "}\OtherTok{\textasciigrave{}renamed\_vars\textquotesingle{}} \OtherTok{\textasciigrave{}vvvcode\textquotesingle{}}\StringTok{"            }
\StringTok{        \}}
\StringTok{    \}}
\StringTok{    file write myfile \textasciigrave{}"}\NormalTok{keep}\OtherTok{\textasciigrave{}renamed\_vars\textquotesingle{}}\StringTok{"\textquotesingle{} }

\StringTok{    // close file}
\StringTok{    file close myfile}
\StringTok{\}}
\end{Highlighting}
\end{Shaded}

\begin{Shaded}
\begin{Highlighting}[]
\KeywordTok{clear} \OtherTok{all}

\KeywordTok{tempfile}\NormalTok{ combined\_data}
\KeywordTok{save} \OtherTok{\textasciigrave{}combined\_data\textquotesingle{}}\NormalTok{, emptyok}

\NormalTok{forval i = 2014 (2) 2018 \{}
    \CommentTok{// Use file corresponding to each year}
    \KeywordTok{use} \StringTok{"$temp/individual\_\textasciigrave{}i\textquotesingle{}.dta"}\NormalTok{, }\KeywordTok{clear}

    \CommentTok{// Run file .do corresponding to each year}
    \KeywordTok{do} \StringTok{\textasciigrave{}"}\OtherTok{$temp}\NormalTok{/label\_in}\OtherTok{\textasciigrave{}i\textquotesingle{}}\NormalTok{.do}\StringTok{"\textquotesingle{}}
\StringTok{    }
\StringTok{    gen year = \textasciigrave{}i\textquotesingle{}}

\StringTok{    // Append data to initially created file}
\StringTok{    append using \textasciigrave{}combined\_data\textquotesingle{}}

\StringTok{    save \textasciigrave{}combined\_data\textquotesingle{}, replace}
\StringTok{\}}

\StringTok{save "}\OtherTok{$temp}\NormalTok{/vhlss\_individual\_14\_18.dta}\StringTok{", replace}
\end{Highlighting}
\end{Shaded}

\section{Inconsistent province codes}\label{inconsistent-province-codes}

As many users of Vietnamese data know, the number of provinces has changed significantly since the late 1980s. In most cases the changing of provincial boundaries was either a splitting or aggregating of existing provinces as opposed to districts being reallocated between provinces. The province codes also change within surveys and across data sources.

All province codes in this processed dataset are consistent across years. The implementation of this standardization can be found in \href{https://sites.google.com/site/briandmccaig/notes-on-vhlsss}{Brian McCaig's website}.

\section{Generate new variables}\label{generate-new-variables}

Below are variables of interest that we create for our research purpose

\subsection*{Education}\label{education}
\addcontentsline{toc}{subsection}{Education}

\begin{Shaded}
\begin{Highlighting}[]
\KeywordTok{gen}\NormalTok{ education = 0 }\KeywordTok{if}\NormalTok{ general\_edu == 0 \& vocational\_edu == 0}
\KeywordTok{replace}\NormalTok{ education = 1 }\KeywordTok{if}\NormalTok{ general\_edu == 1}
\KeywordTok{replace}\NormalTok{ education = 2 }\KeywordTok{if}\NormalTok{ general\_edu == 2}
\KeywordTok{replace}\NormalTok{ education = 3 }\KeywordTok{if}\NormalTok{ general\_edu == 3}
\KeywordTok{replace}\NormalTok{ education = 4 }\KeywordTok{if}\NormalTok{ vocational\_edu == 4}
\KeywordTok{replace}\NormalTok{ education = 5 }\KeywordTok{if}\NormalTok{ vocational\_edu == 5}
\KeywordTok{replace}\NormalTok{ education = 6 }\KeywordTok{if}\NormalTok{ vocational\_edu == 6}
\KeywordTok{replace}\NormalTok{ education = 7 }\KeywordTok{if}\NormalTok{ vocational\_edu == 7}
\KeywordTok{replace}\NormalTok{ education = 8 }\KeywordTok{if}\NormalTok{ general\_edu == 8}
\KeywordTok{replace}\NormalTok{ education = 9 }\KeywordTok{if}\NormalTok{ general\_edu == 9}
\KeywordTok{replace}\NormalTok{ education = 10 }\KeywordTok{if}\NormalTok{ general\_edu == 10}
\KeywordTok{replace}\NormalTok{ education = 11 }\KeywordTok{if}\NormalTok{ general\_edu == 11}
\KeywordTok{replace}\NormalTok{ education = 12 }\KeywordTok{if}\NormalTok{ general\_edu == 12}

\NormalTok{lab }\KeywordTok{var}\NormalTok{ education }\StringTok{"Bằng cấp cao nhất"}
\NormalTok{lab def education\_lbl 0 }\StringTok{"Không có bằng cấp"} \CommentTok{///}
\NormalTok{                  1 }\StringTok{"Tiểu học"} \CommentTok{///}
\NormalTok{                  2 }\StringTok{"THCS"} \CommentTok{///}
\NormalTok{                  3 }\StringTok{"THPT"} \CommentTok{///}
\NormalTok{                  4 }\StringTok{"Sơ cấp nghề"} \CommentTok{///}
\NormalTok{                  5 }\StringTok{"Trung cấp nghề"} \CommentTok{///}
\NormalTok{                  6 }\StringTok{"Trung học chuyên nghiệp"} \CommentTok{///}
\NormalTok{                  7 }\StringTok{"Cao đẳng nghề"} \CommentTok{///}
\NormalTok{                  8 }\StringTok{"Cao đẳng"} \CommentTok{///}
\NormalTok{                  9 }\StringTok{"Đại học"} \CommentTok{///}
\NormalTok{                  10 }\StringTok{"Thạc sĩ"} \CommentTok{///}
\NormalTok{                  11 }\StringTok{"Tiến sĩ"} \CommentTok{///}
\NormalTok{                  12 }\StringTok{"Khác"}
\NormalTok{lab val education education\_lbl}
\end{Highlighting}
\end{Shaded}

\subsection*{Total income}\label{total-income}
\addcontentsline{toc}{subsection}{Total income}

\begin{Shaded}
\begin{Highlighting}[]
\KeywordTok{foreach} \FunctionTok{m} \KeywordTok{in}\NormalTok{ wage\_1 holiday\_bonus\_1 bonus\_1 wage\_2 holiday\_bonus\_2 bonus\_2 \{}
    \KeywordTok{replace} \OtherTok{\textasciigrave{}m\textquotesingle{}}\NormalTok{ = 0 }\KeywordTok{if} \FunctionTok{missing}\NormalTok{(}\OtherTok{\textasciigrave{}m\textquotesingle{}}\NormalTok{)}
\NormalTok{\}}

\KeywordTok{egen}\NormalTok{ income = }\FunctionTok{rowtotal}\NormalTok{(wage\_1 holiday\_bonus\_1 bonus\_1 wage\_2 holiday\_bonus\_2 bonus\_2)}
\KeywordTok{replace}\NormalTok{ income = . }\KeywordTok{if}\NormalTok{ income == 0}

\NormalTok{lab }\KeywordTok{var}\NormalTok{ income }\StringTok{"Tổng thu nhập từ tất cả các nguồn (tiền lương, tiền thưởng, phụ cấp, etc.) trong 12 tháng qua"}
\end{Highlighting}
\end{Shaded}

\subsection*{Food consumption}\label{food-consumption}
\addcontentsline{toc}{subsection}{Food consumption}

\begin{Shaded}
\begin{Highlighting}[]
    \CommentTok{// holiday consumption}
\KeywordTok{egen}\NormalTok{ quant\_h = }\FunctionTok{rowtotal}\NormalTok{(food\_cons1\_q food\_cons2\_q)}
\KeywordTok{egen}\NormalTok{ values\_h = }\FunctionTok{rowtotal}\NormalTok{(food\_cons1\_p food\_cons2\_p)}

    \CommentTok{// daily consumption}
\KeywordTok{ren}\NormalTok{ (food\_cons3\_q food\_cons3\_p) (quant\_d values\_d)}
    
    \CommentTok{// total consumption}
\KeywordTok{foreach} \FunctionTok{m} \KeywordTok{in}\NormalTok{ quant\_d quant\_h values\_d values\_h \{}
    \KeywordTok{replace} \OtherTok{\textasciigrave{}m\textquotesingle{}}\NormalTok{ = 0 }\KeywordTok{if} \OtherTok{\textasciigrave{}m\textquotesingle{}}\NormalTok{ == .}
\NormalTok{\}}

\KeywordTok{gen}\NormalTok{ food\_quant = quant\_d*350/30+quant\_h}
\KeywordTok{gen}\NormalTok{ food\_values = values\_d*350/30+values\_h}

\NormalTok{lab }\KeywordTok{var}\NormalTok{ food\_quant }\StringTok{"Số lượng thực phẩm tiêu thụ (kg)"}
\NormalTok{lab }\KeywordTok{var}\NormalTok{ food\_values }\StringTok{"Trị giá thực phẩm tiêu thụ (nghìn đồng)"}

\KeywordTok{drop}\NormalTok{ quant\_* values\_*}

\KeywordTok{save} \StringTok{"$clean/vhlss\_individual\_14\_18.dta"}\NormalTok{, }\KeywordTok{replace}
\end{Highlighting}
\end{Shaded}

\chapter{Summary Statistics (of processed data)}\label{summary-statistics-of-processed-data}

\section*{Library}\label{library}
\addcontentsline{toc}{section}{Library}

\begin{Shaded}
\begin{Highlighting}[]
\FunctionTok{library}\NormalTok{(tidyverse)}
\FunctionTok{library}\NormalTok{(gt)}
\FunctionTok{library}\NormalTok{(gtExtras)}
\FunctionTok{library}\NormalTok{(summarytools)}
\FunctionTok{library}\NormalTok{(haven)}
\FunctionTok{library}\NormalTok{(sjlabelled)}
\FunctionTok{library}\NormalTok{(webshot2)}
\end{Highlighting}
\end{Shaded}

\section*{Import data}\label{import-data}
\addcontentsline{toc}{section}{Import data}

\begin{Shaded}
\begin{Highlighting}[]
\NormalTok{vhlss }\OtherTok{\textless{}{-}} \FunctionTok{read\_dta}\NormalTok{(}\StringTok{"clean/vhlss\_individual\_14\_18.dta"}\NormalTok{)}

\NormalTok{var\_to\_drop }\OtherTok{\textless{}{-}} \FunctionTok{c}\NormalTok{(}\StringTok{"tinh"}\NormalTok{, }\StringTok{"huyen"}\NormalTok{, }\StringTok{"xa"}\NormalTok{, }\StringTok{"diaban"}\NormalTok{, }\StringTok{"hoso"}\NormalTok{, }\StringTok{"gioitinh"}\NormalTok{, }\StringTok{"dantoc"}\NormalTok{)}
\end{Highlighting}
\end{Shaded}

\section*{Summary statistics}\label{summary-statistics}
\addcontentsline{toc}{section}{Summary statistics}

\begin{Shaded}
\begin{Highlighting}[]
\FunctionTok{source}\NormalTok{(}\StringTok{"script/gt\_summarytools.R"}\NormalTok{)}

\NormalTok{vhlss }\OtherTok{\textless{}{-}}\NormalTok{ vhlss }\SpecialCharTok{\%\textgreater{}\%}
  \FunctionTok{head}\NormalTok{(}\DecValTok{100}\NormalTok{) }\SpecialCharTok{\%\textgreater{}\%} 
  \FunctionTok{select}\NormalTok{(}\FunctionTok{where}\NormalTok{(is.numeric)) }\SpecialCharTok{\%\textgreater{}\%} 
  \FunctionTok{select}\NormalTok{(}\SpecialCharTok{{-}}\FunctionTok{any\_of}\NormalTok{(var\_to\_drop)) }\SpecialCharTok{\%\textgreater{}\%}
  \FunctionTok{mutate}\NormalTok{(}\FunctionTok{across}\NormalTok{(}\FunctionTok{everything}\NormalTok{(), }\SpecialCharTok{\textasciitilde{}} \FunctionTok{ifelse}\NormalTok{(.x }\SpecialCharTok{\textless{}} \DecValTok{0}\NormalTok{, }\ConstantTok{NA}\NormalTok{, .x))) }\SpecialCharTok{\%\textgreater{}\%}
  \FunctionTok{copy\_labels}\NormalTok{(vhlss)}
\end{Highlighting}
\end{Shaded}

\begin{Shaded}
\begin{Highlighting}[]
\FunctionTok{gt\_summarytools}\NormalTok{(}\AttributeTok{data =}\NormalTok{ vhlss, }\AttributeTok{title =} \StringTok{"VHLSS 2014{-}2018 Data Summary"}\NormalTok{)}
\end{Highlighting}
\end{Shaded}

\chapter{Reference}\label{reference}

VHLSS is a popular dataset and has been used by many scholars. Here are some of the notes from other economists that we should read:

\begin{itemize}
\tightlist
\item
  Notes on Vietnamese data by \href{https://sites.google.com/site/briandmccaig/notes-on-vhlsss}{Brain McCaig}
\item
  Data and Rscript sharing by \href{https://sites.google.com/tmu.edu.vn/huongtrinhthi/data-and-rscript-sharing}{Trinh Thi Huong}
\item
  Notes on Vietnamese Microdata by \href{https://rpubs.com/lananhh269/notesonvnmdata}{Lan Anh Ngo}
\item
  Gaps in Household Income From VHLSS 2018 by \href{https://rpubs.com/chidungkt/789118}{Nguyen Chi Dung}
\end{itemize}

  \bibliography{book.bib,packages.bib}

\end{document}
